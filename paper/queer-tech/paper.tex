\documentclass{acmsmall}
%\acmVolume{V}
%\acmNumber{N}
%\acmArticle{A}
\acmYear{2014}
\acmMonth{3}
\markboth{P. Vargas}{Sex, Porn and Computers}
\title{Sex, Porn and Computers: Gender, Sexuality and the need for discussion in the Technology Fields}
\author{Patrick Vargas
   \affil{University of Colorado Boulder}}
\begin{abstract}
Technology has developed greatly since the 20th Century. With the rise of the Internet, Information is more widely available than ever before. Without knowing the end user of technology greatly hinders the advancement. The gender digital divide in the professional computer and information science fields is abysmal. Considering Women are becoming the majority demographic of the Internet. Sexuality, expressed through pornography, is widely profitable and advances technology and yet is not discussed or researched academically. Adolescents discovering their identities, specifically sexual, often times need external literature to develop. Especially for Queer youths, and/or youths in remote, un-accepting environments need the Internet to discover non-heteronormative sexualities and non-binarial genders. Overall there is a need to further the research of the end user of various technologies in order to better development and advance the fields of technology and in essence, humanity in general.
\end{abstract}
\category{J.3.4}{Social and professional topics}{Sexual Orientation}[User Characteristics]
\category{J.3.3}{Social and professional topics}{Gender}[User Characteristics]
% \terms{}
\keywords{Bisexual, Gay, Gender, Human-Centered Computing, Human-Computer Interaction, Lesbian, LGBT, Pornography, Sexuality, Transgender}
\acmformat{Patrick Vargas. 2014. Sex, Porn and Computers.}
\begin{document}
\begin{bottomstuff}
\end{bottomstuff}
\maketitle

\section{Introduction}

The Internet: The final frontier. This piece of technology brings people and information closer together than ever before. The wealth of resources abound are unfathomable. The opportunities to share dialog with people like and unlike ourselves is extraordinary. The idea that a person can discover about them selves from the safety of their home was unimaginable even fifty years ago. Unfortunately there is a down side to this narrative. The fact of the matter is this technology has been one-sided, with it's development dominated by cisgendered males. What's more is, ``[\ldots] information systems research is predominantly, although not exclusively [\ldots], heteronormative in nature.'' \cite{light10}. There are many different kinds of people who use technology than those who create it. Without knowing who uses technology is poor design as developers should always be thinking about the user. There is the need to further the dialog of gender and sexuality in the technology fields of computer science and information systems research in order to further these fields into the future.

\section{The Gender Digital Divide}

First of all we have the gender digital divide. In Science, Technology, Engineering and Mathematics (STEM), we can see that there are significantly less females than males employed. ``A 2011 report by the U.S. Department of Commerce found only one in seven engineers is female. Additionally, women have seen no employment growth in STEM jobs since 2000.'' \cite{huhman12} This is problematic as about 50\% of the world is female. In addition, we know very little about the genetic makeup of those who use Information and Communication Technology (ICT). ``[\ldots P]articularly in developing countries, [\ldots] the data---sex-disaggregated statistics and gender indicators on ICTs---in many cases are not there. Collecting and analyzing data on how ICTs impact men and women differently are a necessary prerequisite to achieving a globally equitable information society.'' \cite{hafkin08} This gap in information overlooks a major user subset. Without this knowledge of who uses what technology hinders the advancement of the field.

So why should the STEM fields focus on this gender gap? Well, women are now dominating the Internet. Burst Media \citeyear{burst13} has found the majority of Internet users are social media users. ``Two-thirds (65.4\%) of respondents have at least one personal social media account from providers including Facebook, Google+ and Twitter---and 3-in-5 (58.6\%) use their account(s) at least once a day'' \cite{burst13} In addition, women use these networks more than men do. According to Burst Media \citeyear{burst13}, ``More women than men (56.0\% versus 49.5\%) have a Facebook account. [\ldots] One-fifth (21.9\%) of female respondents have a Pinterest account, 
versus only 4.8\% of men. [\ldots] Instagram's audience also skews more female than male---10.4\% versus 5.8\%, respectively.'' Clearly, not giving attention to gender and the digital divide is detrimental both socially and economically. With females as the dominate user of the Internet, the computer and information science fields can no longer overlook the gender gap.

\section{The Elephant in the Room}

Sexuality is one topic the computer and information science fields also neglect. No one talks about it as many times it's considered taboo. Yet this one side of the Internet, of which during the Dot Com crash of the late Nineties raked in \$1 Billion in 1998 \cite{branwyn99}, is the most prolific. Indeed, this field is represented often times as pornography. ``The production and consumption of pornography are some of the most popular and commercially successful applications of digital technology in the world. Porn and sex have been the most frequent Internet search terms since the Web became widely available.'' \cite{bell05} This is a major part of the Internet and of the human experience, of which the Computer and Information Science fields neglects to discuss and include. ``The topic of sexuality has generally been mythicized as a taboo due to various reasons in varying degrees. While there is nothing intrinsic to the topic that makes it difficult to talk about, `once a subject is tabooed, that status begins to feel self-evident'.'' \cite{kannabiran11} This stigma in the academic world around sexuality is difficult to shake. Many scholars fear what colleagues and fellows would think if they researched this part of computer and information science. Others are unwilling to shake their own prejudices around sexuality and pornography.

This hurdle the Human-Computer Interaction (HCI) fields deals with is difficult and burdensome. On the one side, without exploring this widely successful field of technology is detrimental socially and economically, as stated above. In addition, pornography leads the way in the development and advancement of technology. ``[\ldots C]onsumers of pornography have accelerated the diffusion of new communication technologies like the VCR and CD-ROM by becoming early buyers and users, thereby providing a profitable market for newly introduced services. Their willingness to pay an initial premium increased early sales, thus reducing costs for later buyers who benefited from the economies of larger markets for more mainstream services.'' \cite{coopersmith99} Technologies like the switch from Beta to VCR was fueled by the need to consume pornography. Some titles at the time were only available on VHS. This lead to the adoption of the VCR as people felt the need to consume such materials.

In addition, the use of pornography is staggering. ``After an analysis of 400 million web searches from July 2009 to July 2010, researchers concluded [\ldots] 13\% of all searches were for erotic content.'' \cite{eyes13} Porn is also very profitable. ``In 2006, the sex-related entertainment business' estimated revenues were just under \$13 billion in the U.S. [\ldots] In 2007, global porn revenues were estimated at \$20 billion, with \$10 billion in the U.S. [\ldots I]n 2005 pornography accounted for 69\% of the total pay-per-view Internet content market, outpacing news, sports, and video games.'' \cite{eyes13} This is one elephant that just can't be ignored forever.

Lastly, the specific demographic data is limited. ``A 2001 Forrester Research report claimed the average age of a male visitor to an adult web page was 41 and had an annual income of \$60,000. According to the same report, 19\% of North American users were regular visitors to adult content sites. Of that 19\%, approximately 25\% were women, 46\% were married, and 33\% had children.'' \cite{eyes13} This study does not include race or sexuality. 

\section{Queer Youth and Self Discovery}

There is one group that is often marginalized even more so than other minorities in this regard. The Lesbian, Gay, Bisexual and Transgender (LGBT) population often is overlooked. Yet the fact of the matter is this community uses technology, specifically the Internet, to shape who they are as they grow and develop in adolescence. ``[T]he Internet plays an important role in early adolescents' declaration of sexuality because it provides information that allows teens to label feelings and figure out who they are. [\ldots Y]oung males in the pre-coming out stage used Internet pornography to understand and develop their same-sex feelings.'' \cite{bond13} If HCI professionals continue to think of the end user as a heteronormaitve, cisgendered male, then this isolates the user that is not of that disposition. This isolation negatively impacts the way the user uses the technology by forcing the user to recognize that they are not part of the majority.

Take for example a virtual world that involves avatar creation. ``Gender remains binary in virtual worlds. One is either a male or a female; no in–between is possible.'' \cite{blodgett07} This binary forces the person to choose, making the user realize once again that they might not fit within the traditional, gender binary that society has constructed.  ``Since these limitations often are hard-coded into the software, they represent the solidifying of social stereotypes into the structures of the virtual world. Several virtual worlds allow players to marry each other but only if the avatar couples consist of heterosexual male and female.'' \cite{blodgett07} This forced relationship only further isolates the the player and may cause them to discontinue play.

Youths of all origins go through this stage of self identification of sexual identity and feelings. Without this stage, adolescents can spiral into an identity crisis. ``When an adolescent does not understand his or her identity, sexuality included, the individual has potential to enter a stage known as role confusion [\ldots]. In this stage, a young person enters a `moratorium', eventually leading to an identity crisis that can be detrimental to the overall development of the adolescent's identity. To avoid role confusion and eventual identity crisis, it becomes important for adolescents to develop strong relationships with peers.'' \cite{bond13} This confusion not only impacts the person but also those around them. If an individual doesn't know who they are, they may try to fit into someone else's predefined mold. This person will not seem right in this role and won't be happy or content. Finally, when the individual does come to terms with their identity, the longer they were in a predefined role, the more disastrous the outcome could be.

Today, this self discovery comes mainly from the Internet and other social technology. ``[P]articipation in Internet newsgroups during sexual identity formation has led to greater self-acceptance and disclosure of hidden sexual identity to family and friends [\ldots]. These examples illustrate how the Internet may serve as a catalyst to push individuals from the pre-coming out stage to the coming-out stage.'' \cite{bond13} This mediated environment helps queer youths to develop and learn who they are in a safe environment. 

Sadly, many young gay people are often isolated in remote areas, where often they are the only queer person they know. This isolation can be damaging to the person in question psychologically, especially in inhospitable environments filled with intolerance. ``  [\ldots] LGB adolescents' awareness of social norms vilifying their attraction to same-sex individuals may lead many to suppress their sexual identity from their established peer networks for fear of being ostracized[. \ldots It's been] reported that 95\% of gay youth surveyed expressed feelings of isolation from peers. If LGB adolescents are not experiencing sexual identity formation with the assistance of face-to-face communication among peers, they may look elsewhere for support and information.'' \cite{bond13} Thus, we as computer and information research professionals need to bring this minority exclusion to light. Our widespread lack of apathy to this community has wide reaching effects to the development of queer youth.

\section{Conclusion}
The Internet is here to stay and is only becoming more and more wide reaching. In addition, more technology is being developed and adopted to bring people together. However, without knowing the genetic makeup of the end users is detrimental. The disproportionate gender gap in professional computing is horrendous. With women slowly becoming the majority of users of the Internet, there needs to be more females developing the technology as well as being studied for behavioral use studies. 

Sexuality, and specifically pornography, can no longer be taboo and considered non-academic. This widely successful media not only has proven to be widely profitable, but also the considering factor of technology adoption. In addition, more research needs to be done one the demographics of the end-user. We still don't know the demographic information of the end-users concerning race and sexuality.

Furthermore, adolescents of any sexuality and gender use technology in their self-identification. Especially queer identified youths and youths in remote, un-accepting environments need access to the Internet and information in general about alternative and non-heteronormitve genders and sexualities. We as computer and information science professionals need to take further action to identify the wide user set of technologies to better advance the fields. Without knowing how developer's personal opinions impact the end user is detrimental, and may even be costly. We no longer have the luxury to interpret the end user as a heternormative, cisgender male. The time of the one-size-fits-all computing model is over and the age of the consumer is in full swing. For the betterment of humanity, we need to must research and reach out to humanity.

\bibliographystyle{ACM-Reference-Format-Journals}
\bibliography{paper}

\end{document}